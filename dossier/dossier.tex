\documentclass[a4paper]{article}

\usepackage[utf8x]{inputenc}
\usepackage[francais]{babel}
\usepackage[T1]{fontenc}
\usepackage{amsmath}
\usepackage{graphicx}
\usepackage{hyperref}
\usepackage[colorinlistoftodos]{todonotes}
\usepackage[parfill]{parskip}

\title{Aide à la décision multicritères \\ Rapport de projet}
\author{\\\\ Tristan DANIEL, Adrien ERCKSEN, Pierre FERNANDEZ,\\ Tanguy MAZE, Quentin MAZOUA, Loïc REDON \\\\ Encadrant : Pierre NERZIC \\\\\\\\\\\\\\\\ IUT de Lannion}

\begin{document}
\maketitle

\newpage

\subsection*{Introduction}

\paragraph{Contexte}
Nous sommes six étudiants en seconde année de DUT informatique à l'IUT de Lannion. Dans le cadre de notre formation, nous avons été amenés à réaliser un projet. Celui-ci n'a pas pour unique but de programmer ; tous les aspects et acteurs de la conception d'une application sont représentés. Ainsi, plutôt que de se limiter à appliquer les consignes et à en tirer une application générique, nous avons été amenés à effectuer des recherches, expérimenter et prendre des décisions concernant les outils à utiliser.

\paragraph{Le choix du sujet}
Deux choix s'offraient à nous : sélectionner un sujet parmi ceux proposés ou en proposer un personnalisé. Nous avons étudié les sujets mis à notre disposition, et notre intérêt s'est porté sur celui concernant l'aide à la décision multicritères. Ce choix a été motivé par plusieurs raisons. Premièrement, le fait que celui-ci concerne les méthodes d'aide à la décision ; il nous a paru intéressant d'en apprendre un peu plus sur celles-ci. Ces méthodes peuvent s'avérer utiles dans de nombreuses situations du quotidien. Deuxièmement, le fait que l'application elle-même ne concerne pas l'informatique pure : le travail de recherche à effectuer et les formules mathématiques et algorithmes de comparaison mis en oeuvre par ces méthodes constituent un socle théorique au projet.

\paragraph{But du projet}
Le but est de produire une application employant les méthodes ELECTRE qui puisse fournir une aide à la décision dans le cadre de choix parmis plusieurs options aux critères communs et multiples. Nous décrirons dans ce document certaines des méthodes existantes, leurs avantages et leurs inconvénients.

\newpage
\tableofcontents

\newpage

\section{La décision multicritères}

\subsection{Principes}

L'aide à la décision est une approche scientifique permettant à l'utilisateur d'effectuer le meilleur choix possible lui correspondant. L'utilisation d'algorithmes de calcul assure l'obtention rapide de résultats de tri corrects sans les erreurs humaines possibles dans le cas de classements à la main ou par simple jugement.

\paragraph{Candidat} On appelle \textit{candidat}, \textit{solution} ou \textit{action}, un élément que l'utilisateur souhaite classer.

\paragraph{Critère} Un \textit{critère} est, comme son nom l'indique, une propriété partagée par tous les candidats sur laquelle ils seront classés. Chaque candidat doit attribuer une valeur numérique à chaque critère. Un critère possède un \textit{poids} qui détermine son importance. Un poids négatif peut être utilisé si l'on veut minimiser un critère tel qu'un prix.

\par
Les algorithmes de tri fonctionnent selon un principe commun : pour chaque candidat, on calcule un ou plusieurs scores qui servent ensuite à classer les candidats du plus au moins intéressant ou par groupes rassemblant plusieurs candidats proches.

\newpage

\subsection{Méthode naïve}

La méthode naïve (ou méthode classique) de tri multicritères consiste à attribuer un score total à une solution qui est la somme des valeurs de chaque critère. Soit \(C\) un candidat, \(C_i\) la valeur pour le critère \(i\), \(p_i\) le poids du critère \(i\) et \(n\) le nombre de critères ; le score d'un candidat est calculé par somme pondérée :
\[\mbox{score}_{C}=\sum_{i=1}^nC_i*poids_i\]

Cette méthode pose cependant plusieurs problèmes, dont le plus gênant est l'effet de compensation. En effet, il suffit qu'un candidat ait des scores excellents dans assez de critères à bas coefficients pour compenser un mauvais score dans un critère important.

\begin{table}[!ht]
\begin{tabular}{|c|c|c|c|c|}
  \hline
  \hphantom & Coeurs CPU & Fréquence CPU & Espace SSD &  Espace HDD \\
  \hline
  Ordinateur 1 & 4 & 3.5 & 128 & 3000 \\
  Ordinateur 2 & 8 & 3 & 256 &	2000 \\
  Ordinateur 3 & 2 & 3 & 0 & 750 \\
  \hline
\end{tabular}
\caption{Liste de candidats, ici des ordinateurs}
\end{table}

\newpage

\subsection{Méthodes ELECTRE}

Les méthodes ELECTRE sont des méthodes d'aide à la décision multicritères dont la première (ELECTRE I) a été imaginée par Bernard ROY dans les années 1960. Son but était de représenter les choix de la façon la plus réaliste possible (d'où ELECTRE : ELimination Et Choix Traduisant la REalité) afin d'aider les entreprises dans la prise de décision. Depuis, de nombreuses méthodes dérivées sont apparues avec chacune leurs avantages (ELECTRE II, ELECTRE III, ELECTRE TRI, etc.). Dans ce projet, nous nous intéresseront à la méthode ELECTRE I.

\subsubsection{Méthode ELECTRE I}

Première méthode ELECTRE, elle reprend les poids de critères de la méthode classique. Cependant, toute ressemblance entre les deux méthodes s'arrête ici car, contrairement à la méthode pondérée, la méthode ELECTRE va comparer les performances de tous les candidats entre eux pour déterminer deux coefficients : celui de concordance et celui de discordance. Ceux-ci serviront à déterminer le surclassement (supériorité) d'un candidat par rapport à un autre.

\paragraph{Concordance}
Soient deux solutions \(a\) et \(b\) ; \(a\) est concordante à \(b\) si la somme des poids des critères pour lesquels la performance de \(a\) est supérieure ou égale à celle de \(b\) sur la somme totale des poids des critères est inférieure à un seuil de concordance défini par l'utilisateur. Dans ce cas, la solution \(a\) est considérée comme équivalente à \(b\) et ne la surclasse pas.

\paragraph{Discordance}
Soient deux solutions \(a\) et \(b\), \(a\) est discordante à \(b\) si la plus grande différence de performance entre tous les critères de \(a\) et \(b\) sur la plus grande différence de performance observée entre toutes les solutions est supérieure à un seuil de discordance défini par l'utilisateur. Dans ce cas, la solution \(a\) est considérée comme totalement différente de \(b\), si bien qu'aucune ne surclasse l'autre.

\newpage

\section{Projet}

\subsection{Architecture}

Le projet à réaliser est une application qui doit permettre à un utilisateur n'ayant pas connaissance des méthodes citées plus haut de réaliser sa propre comparaison. Cette application doit être composée d'un logiciel client ainsi que d'un moteur de calcul indépendant.

\subsubsection{Client}

\paragraph{Projets} Le logiciel client fonctionne sur la base de \textit{projets}, qui représentent des instances de l'application. L'utilisateur peut ainsi créer un projet, l'enregistrer sous forme de fichier sur son ordinateur, l'ouvrir depuis un fichier ou laisser l'application se charger de l'enregistrer automatiquement dans le navigateur.

\paragraph{Critères} Le logiciel doit permettre à l'utilisateur de définir ses propres critères de classement. Il existe deux types de critères : les critères simples dont la valeur est numérique, et les critères spéciaux, non-numériques, dont la valeur est choisie dans une liste prédéfinie (par exemple, un utilisateur veut entrer une couleur comme critère spécial, il créera ce critère en sélectionant les couleurs existantes et leur attribuera une valeur : 1 pour le bleu et 2 pour le vert par exemple ; le vert sera donc préféré au bleu). Lorsque l'utilisateur crée un critère, une colonne est ajoutée dans le tableau des candidats. 

\paragraph{Candidats} L'utilisateur peut ensuite saisir la liste des candidats, avec une valeur pour chaque critère, puis soumettre les données au moteur de calcul.

\subsubsection{Moteur de calcul}

Le moteur de calcul est appelé par le logiciel client afin d'appliquer la méthode ELECTRE. Il doit cependant rester autonome et pouvoir être appelé en ligne de commande. De ce fait, il peut être considéré comme une application \textit{serveur}. Par commodité, cette application serveur est également chargée de transmettre la page web client au navigateur de l'utilisateur.

\newpage

\subsection{Choix des technologies}

\subsubsection{Client}

Nous avons choisi de réaliser l'interface client sous forme de page web, afin de ne pas nécessiter d'installation sur l'ordinateur de l'utilisateur. De plus, les langages Web permettent de réaliser des interfaces graphiques très personnalisées. Nous avons donc opté pour les langages HTML, CSS et JavaScript.

\paragraph{Base} Le framework Bootstrap fournit tous les composants de base et de mise en page nécéssaires à la création de l'application. Il permet de créer des pages web \textit{responsive} et esthétiques rapidement. 

\paragraph{Saisie des candidats} Afin de permettre aux utilisateurs de saisir la liste des candidats, l'application doit contenir un tableau dans lequel il est possible d'ajouter, modifier et supprimer des lignes et des colonnes. Les tableaux HTML de base ne permettent pas cela, il faut donc avoir recours à une librairie JavaScript tierce. Malgré un large choix, il a été difficile de trouver un système existant répondant à nos critères. Le plugin jsGrid pour jQuery est la solution qui se rapproche le plus de nos besoins.

\subsubsection{Serveur}

La partie serveur doit être capable d'interagir avec la page web client. Le seul langage permettant cela que nous connaissions était PHP, et nous l'avons donc choisi. PHP convient parfaitement car il comporte de nombreuses fonctionnalités de traitement de données, facilitant ainsi la programmation des algorithmes ELECTRE.

\subsubsection{Communication client - serveur}

\paragraph{Protocole} La voie de communication la plus simple d'utilisation avec les langages choisis est le protocole HTTP. Celui-ci contient lui-même plusieurs méthodes, dont GET et POST qui permettent de transférer des informations. La méthode POST a été choisie car elle permet de transférer de plus grandes quantités de données que la méthode GET.

\paragraph{Format} Afin de pouvoir communiquer, le client et le serveur doivent échanger des données dans un format commun. Le plus simple d'utilisation avec le client est le JSON, un format créé dans le but de représenter un objet JavaScript sous forme de chaîne de caractères. PHP est également capable d'interpréter ce format à l'aide de fonctions des librairies standards. 

\newpage

\subsection{Implémentation}

\subsubsection{Problèmes rencontrés}

La principale source de problèmes a été le tableau dynamique. Il est difficile de trouver une solution existante qui réponde à tous les critères que nous nous étions fixés. Réaliser ce système nous-mêmes aurait été plus rapide et plus flexible.

\subsubsection{Etat d'avancement}

Voici les fonctionnalités qui ont pour l'instant été implémentées :

\begin{itemize}
\item Chargement de la page web client depuis le serveur
\item Création, sauvegarde et suppression de projets
\item Création de critères simples et spéciaux
\item Ajout, modification et suppression de lignes
\item Export des données en format CSV
\item Traitement des données par le serveur pour la méthode ELECTRE I
\item Affichage des résultats retournés par le serveur
\end{itemize}

\subsubsection{Améliorations possibles}

\paragraph{Fonctionnalités} Notre implémentation ne supporte actuellement que la méthode ELECTRE I. Il serait intéressant d'implémenter ELECTRE III et Tri car les résultats sont meilleurs.

\paragraph{Ergonomie} Les résultats sont actuellement retournés à l'utilisateur sous forme de numéros de candidats. Il faudrait les afficher dans un tableau séparé pour une meilleure lisibilité et éviter toute confusion.

\paragraph{Performances} La sauvegarde en particulier peut engendrer des problèmes de performance. Actuellement, à chaque modification de critère ou de donnée, une sauvegarde complète est effectuée. Cela ne pose pas de problèmes sur des petits volumes, mais si l'utilisateur travaille sur de gros fichiers, il pourrait subir des ralentissements.

\paragraph{Interface} L'interface client n'est pour le moment pas \textit{responsive}. Même s'il est peu probable qu'un utilisateur saisisse des grandes quantités de données depuis un terminal mobile, c'est un cas qu'il faut envisager.

\subsubsection{Résultat}

Le code source final est hébergé à l'adresse \url{https://github.com/MazouaIndustries/dms}, et une démonstration a été mise en ligne sur \url{http://dms.13h37.tk}.

\newpage

\subsection{Organisation}

\subsubsection{Répartition des tâches}

\begin{itemize}
\item \textbf{Quentin} : interface graphique, gestion des critères et colonnes
\item \textbf{Tristan} : serveur PHP, requêtes Ajax
\item \textbf{Adrien} : Recherches (JSON, tableaux dynamiques)
\item \textbf{Pierre} : import - export de projets
\item \textbf{Tanguy} : algorithme ELECTRE I
\item \textbf{Loïc} : import - export des données au format CSV, aide utilisateur
\end{itemize}

Le gantt prévisionnel est disponible à l'adresse
\url{http://dms.13h37.tk/gantt.html}.

\subsubsection{Méthode de développement}

Tout au long du processus de développement de l'application, nous avons utilisé la plateforme GitHub pour versionner notre code. Ainsi, tout ajout de fonctionnalité ou correction de bug pouvait être partagé facilement et rapidement avec les autres membres du groupe.

\end{document}
